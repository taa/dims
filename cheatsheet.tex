\documentclass[a4paper]{article}
\usepackage[utf8]{inputenc}
\usepackage{amssymb, amsmath}
\usepackage[pdftex]{graphicx}
\usepackage{pdfpages}
\usepackage[colorlinks=true,linkcolor=black]{hyperref}
\usepackage{float}
\usepackage[all]{xy}


\newcommand{\Integers}{\mathbb{Z}}

\renewcommand{\labelenumi}{(\arabic{enumi})}
\renewcommand{\labelenumii}{(\Alph{enumii})}

\title{Diskrete matematiske strukturer\\
          Cheat sheet}
\author {}
\date{}

\begin{document}
\maketitle
\tableofcontents
\clearpage
\section{Mængder og sæt}
  Indhold
  \subsection{Intersection}
    Indhold
  \subsection{Union}
    Indhold
  \subsection{Complements}
    Indhold
  \subsection{Andre regler}
    Indhold
\section{Heltal}
  Indhold
  \subsection{GCD}
    If $a$, $b$ and $k \in \Integers^+$, and $k | a$ and $k | b$, then $k$ is a common divisor of $a$ and $b$. If $d$ is the largest of such $k$ it's called the greatest common divisor. Denoted:
    $$d = GCD(a, b)$$
    $d$ is the greatest common divisor of $a$ and $b$ and is also a multiple of all lesser divisors.
    \subsubsection{Euklids algoritme}
     Euklids algoritme is a way to find the greatest common divisor and is easiest illustrated by example.\\
     Theory: Integer divide $a$ by $b$ until the rest is zero.\\
     Example: GCD(273, 98). Let $a$ be 273 and $b$ be 98.
     \begin{eqnarray}
      divide\ 273\ by\ 98:	273 = 2 \cdot 98 + 77\\
      divide\ 98\ by\ 77:	98 = 1 \cdot 77 + 21\\
      divide\ 77\ by\ 21:	77 = 3 \cdot 21 + 14\\
      divide\ 21\ by\ 14:	21 = 1 \cdot 14 + 7\\
      divide\ 14\ by\ 7:	14 = 2 \cdot 7 + 0
     \end{eqnarray}
     The above concludes GCD(98, 77) = 7
  \subsection{LCM}
   If $a$, $b$ and $k$ are $\in \Integers^+$, and $a | k$ and $b | k$, then $k$ is a common multiple of $a$ and $b$. The smallest such $k$, is called the least common multiple. Denoted:
    $$c = lcm(a, b)$$
  \subsection{Base$_n$}
    Indhold
  \subsection{Primfactoresering}
    Indhold
\section{Matricer}
  Indhold
  \subsection{Produkt}
    Indhold
  \subsection{Sum}
    Indhold
  \subsection{Boolske matricer}
    En boolsk matrix er en matrix med alfabetet ${0;0}$. Fx:
    $\left[\begin{smallmatrix}
      1 & 0 & 1 \\
      0 & 1 & 0 \\
      1 & 0 & 0  \\
    \end{smallmatrix}\right]$
    \subsubsection{OR}
      OR mellem to boolske matricer gøres på følgende måde:
      Lad $A = [a_{ij}]$ og $B= [b_{ij}]$ være $m \times n$ boolske matricer. Så er:
      $$
        C = c_{ij} = \begin{bmatrix}
                    a_{1,1} \vee b_{1,1} & \cdots & a_{1,n} \vee b_{1,n} \\
                    \vdots               & \ddots & \vdots \\
                    a_{m,1} \vee b_{m,1} & \cdots & a_{m,n} \vee b_{m,n}
                  \end{bmatrix}
      $$
      ved $A \vee B = C$.
    \subsubsection{AND}
      AND mellem to boolske matricer gøres på følgende måde:
      Lad $A = [a_{ij}]$ og $B= [b_{ij}]$ være $m \times n$ boolske matricer. Så er:
      $$
        C = c_{ij} = \begin{bmatrix}
                    a_{1,1} \wedge b_{1,1} & \cdots & a_{1,n} \wedge b_{1,n} \\
                    \vdots               & \ddots & \vdots \\
                    a_{m,1} \wedge b_{m,1} & \cdots & a_{m,n} \wedge b_{m,n}
                  \end{bmatrix}
      $$
      Ved $A \wedge B = C$.
    \subsubsection{Boolsk produkt} %TODO
      Boolsk produkt $(\odot)$ mellem to boolske matricer gøres på følgende måde:
      Lad $A = [a_{ij}]$ og $B= [b_{ij}]$ være $m \times n$ boolske matricer. Så er:
      Ved $A \odot B = C$.
\section{Logik}
  Indhold
  \subsection{Regneregler}
    \subsubsection{Boolske udtryk}
      \begin{description}
        \item[$\sim$ (not)] Bla bla.
        \item[$\wedge$ (and)] Bla bla.
        \item[$\vee$ (or)]  Bla bla.
        \item[$\Rightarrow$ (If..then)]  Bla bla.
        \item[$\Leftrightarrow$ (Equivalence)]  Bla bla.
      \end{description}
      \begin{center}
        \begin{tabular} {|r|r||r|r|r|r|r|r|}
          \hline
          $p$ & $q$  & $\sim{}p$ &  $p\wedge{}q$ &  $p\vee{}q$ &  $p\downarrow{}q$  &  $p\Rightarrow{}q$  &  $p\Leftrightarrow{}q$ \\
          \hline
          \hline
          S&S&F&S&S&F&S&S\\
          \hline
          S&F&F&F&S&F&F&F\\
          \hline
          F&S&S&F&S&F&S&F\\
          \hline
          F&F&S&F&F&S&S&S\\
          \hline
        \end{tabular}
      \end{center}
    \subsubsection{Sætninger og regler}
      Grundregler:
      \begin{eqnarray}
        p \vee q &\equiv& q \vee p \\
        p \wedge q &\equiv& q \wedge p \\
        p \vee (q \vee r) &\equiv& (p \vee q) \vee r \\
        p \wedge (q \wedge r) &\equiv& (p \wedge q) \wedge r \\
        p \vee (q \wedge r) &\equiv& (p \vee q) \wedge (p \vee r) \\
        p \wedge (q \vee r) &\equiv& (p \wedge q) \vee (p \wedge r) \\
        p \vee p &\equiv& p \\
        p \wedge p &\equiv& p \\
        \sim (\sim p) &\equiv& p \\
        \sim (p \vee q) &\equiv& (\sim p) \wedge (\sim q) \\
        \sim (p \wedge q) &\equiv& (\sim p) \vee (\sim q) 
      \end{eqnarray}
      Vigtige sætninger:
      \begin{eqnarray}
        (p\Rightarrow q) &\equiv& ((\sim p) \vee q) \\
        (p \Rightarrow q) &\equiv& (\sim p \Rightarrow \sim q) \\
        (p \Leftrightarrow q) &\equiv& ((p \Rightarrow q) \wedge (q \Rightarrow p)) \\
        \sim (p \Rightarrow q) &\equiv& (p \wedge \sim q) \\
        \sim (p \Leftrightarrow q) &\equiv& ((p \wedge \sim q) \vee (q \wedge \sim p))
      \end{eqnarray}
      Flere vigtige sætninger:
      \begin{eqnarray}
        \sim (\forall x~P(x)) &\equiv& \exists x~\sim P(x) \\
        \sim (\exists x~P(x)) &\equiv& \forall x~(\sim P(x)) \\
        \exists x~(P(x) \Rightarrow Q(x)) &\equiv& \forall x~P(x) \Rightarrow \exists x~Q(x) \\
        \exists x~(P(x) \vee Q(x)) &\equiv& \exists x~P(x) \vee \exists x~Q(x) \\
        \forall x~(P(x) \wedge Q(x)) &\equiv& \forall x~P(x) \wedge \forall x~Q(x)
      \end{eqnarray}
  \subsection{Metoder for beviser}
    Indhold

\section{Sandsynlighedsregning}

\subsection{Permutationer og kombinationer}
En begivenhed er {\it ordnet}, hvis rækkefølgen af udtagelsen har betydning. Den er
{\it med tilbagelægning}, hvis det samme objekt kan vælges flere gange.

\begin{figure}[h!]
    \begin{center}
    \begin{tabular}{|c|c|c|}
        \hline
        & Uden tilbagelægning & Med tilbagelægning \\
        \hline
        Ordnet  & $_nP_r$ & $n^r$ \\
        \hline
        Uordnet & $_nC_r$ & $_{n+r-1}C_r$ \\
        \hline
    \end{tabular}
    \end{center}
    \caption{Oversigt over formler for permutationer og kombinationer} \label{fig:perm}
\end{figure}

hvor $_nP_r$ og $_nC_r$ er defineret ved

\begin{eqnarray*}
    _nP_r &=& n\cdots{}(n-1)\cdots{}(n-r+1) = \frac{n!}{(n-r)!} \\
    _nC_r &=& \frac{n\cdots (n-1) \cdots (n-r+1)}{r\cdot (r-1)\cdots 1} = \frac{n!}{r!(n-r)!}
\end{eqnarray*}

\begin{description}

\item[Multiplikationsprincippet] siger, at hvis en begivenhed $A$ kan forekomme på $m$ forskellige
måder, og en anden, uafhængig begivenhed $B$ kan forekomme på $n$, så er det totale antal måder
hvorpå de to begivenheder kan ske lig $mn$. E.g. hvis du har tre bluser og to bukser har du
$3 \cdot 2 = 6$ muligheder.

\item[Additionsprincippet] siger, at hvis en begivenhed $A$ kan ske på $m$ måder, og en anden,
uafhængig begivenhed $B$ kan ske på $n$ måder, så er der $m + n$ måder som {\it enten} $A$ eller $B$
kan ske på.

\end{description}

\subsection{Sandsynlighed}
Indhold

\section{Lineære homogene rekurrensligninger}

For at bestemme et direkte (ikke-rekursivt) udtryk for en lineær homogen rekurrensligning
af formen $$a_n = r_1a_{n - 1} + r_2a_{n - 2}$$ med to værdier fastlagt som $a_1$ og $a_2$,
kan følgende opskrift benyttes.

\begin{enumerate}

\item Opstil den karakteristiske ligning $$x^2 = r_1x + r_2$$

\item Find rødderne i dette andengradspolynomium ved hjælp af diskriminanten $d$.
\label{homorecurrence:roots}

\begin{enumerate}
\item Hvis $d > 0$, således at der er to forskellige rødder $s_1$ og $s_2$, så vil der
findes tal $u$ og $v$ således at $$a_n = us_1^n + vns_2^n$$

\item Hvis $d = 0$, således at der er én rod $s$, så vil der findes tal $u$ og $v$ således
at $$a_n = us^n + vns^n$$

\item Hvis $d < 0$, således at der ikke er nogen reelle rødder, så kan man benytte komplekse
tal.
\end{enumerate}

\item Indsæt begyndelsesværdierne $a_1$ og $a_2$ i den relevante formel fra (\ref{homorecurrence:roots})
og opstil et lineært ligningssystem med to ligninger og to ubekendte $u$ og $v$.

\item Løs dette ligningssystem og indsæt $u$ og $v$ i den relevante formel fra
(\ref{homorecurrence:roots}).

\end{enumerate}

\section{Relationer}

\subsection{Repræsentation}
 En relation kan repræsenteres på mange måder. Herunder vises relationen $R$ på mængden $\{ 1,2,3,4 \}$:\\
  Repræsenteret som ...formel/udtryk?
  \begin{equation}
    xRy \iff x | y
  \end{equation}
  Repræsenteret som et sæt af ordnede par:
  \begin{equation}
    \{ (1,1),(1,2),(1,3),(1,4),(2,2),(2,4),(3,3),(4,4) \}
  \end{equation}
  Repræsenteret som graf:
  \begin{figure}[H]
    \entrymodifiers={++[o][F-]}
    $$
    \xymatrix @-0pc {
      1 \ar@(l,u) \ar[r] \ar[d] \ar[dr] & 2 \ar@(r,u) \ar[d] \\
      3 \ar@(l,d) &  4 \ar@(r,d)}
      $$
    \caption{Relation som graf} \label{fig:xy1}
  \end{figure}
  Repræsenteret som boolsk matrix:
  \begin{figure}[H]
    $$
    \begin{bmatrix}
      1 & 1 & 1 & 1 \\
      0 & 1 & 0 & 1 \\
      0 & 0 & 1 & 0 \\
      0 & 0 & 0 & 1
    \end{bmatrix}
    $$
    \caption{Relation som boolsk matrix} \label{fig:bm1}
  \end{figure}
\subsection{Egenskaber}
  Herunder noget om de egenskaber relationer kan have.
  \subsubsection{Refleksiv og irrefleksiv}
    En relation $R$ på $\mathbb{R}$ er refleksiv hvis:
    \begin{equation}
      \forall x \in \mathbb{R}: xRx
    \end{equation}
    Det kan også ses i grafen, hvis alle elementer har pile til sig selv, som i figur (\ref{fig:xy1}).

    Det kan også ses på matrixform, hvis diagonalen har 1-taller, som i figur (\ref{fig:bm1}).
\subsection{$R^\infty$}

$R^\infty$ består af alle tupler i $R$, der er transitivt relateret; dvs. hvis $(x, y) \in R$
og $(y, z) \in R$, så vil $(x, z) \in R^\infty$. Derudover gælder $R \subseteq R^\infty$.

\section{Kapitel 4 og frem..}

\end{document}
